\documentclass[11pt]{article}
\usepackage{amsmath,amsfonts,amssymb,amsthm}
\usepackage{times}
\usepackage{graphicx}
\usepackage[ruled,vlined]{algorithm2e}
\usepackage{hyperref}
\hypersetup{
    colorlinks=true,
    linkcolor=blue,
    filecolor=magenta,
    urlcolor=cyan,
}

\urlstyle{same}

\usepackage{geometry}
\geometry{verbose,tmargin=2cm,bmargin=2.5cm,lmargin=2.5cm,rmargin=2.5cm}

\makeatletter

\newtheorem{theorem}{Theorem}[section]
\newtheorem{lemma}[theorem]{Lemma}
\newtheorem{corollary}[theorem]{Corollary}
\newtheorem{claim}[theorem]{Claim}

\theoremstyle{definition}
\newtheorem{definition}[theorem]{Definition}
\theoremstyle{definition}
\newtheorem{example}[theorem]{Example}

\makeatother


%%% fill in details here
\def \lecturedate {August 15, 2020}
%%%

\newcommand{\todo}[1]{\textbf{\textcolor{red}{TODO: #1}}}

%\def\pinv{\dag}

\DeclareMathOperator*{\argmin}{\arg\min}
\DeclareMathOperator*{\argmax}{\arg\max}


\def\X{{\mathcal X}}
\def\H{{\mathcal H}}
\def\Y{{\mathcal Y}}
\def\D{{\mathcal D}}
\def\L{{\mathcal L}}
\def\F{{\mathcal F}}
\def\mP{{\mathcal P}}
\def\reals{{\mathbb R}}
\def\R{{\mathcal R}}
\def\fhat{{\ensuremath{\hat{f}}}}
\def\mH{\mathcal{H}}
\newcommand{\at}{\makeatletter @\makeatother}

\DeclareMathOperator{\clip}{clip}
\def\norm#1{\mathopen\| #1 \mathclose\|}% use instead of $\|x\|$

\newcommand{\proj}{\mathop{\Pi}}

\newcommand{\err}{\mathop{\mbox{\rm error}}}
\newcommand{\rank}{\mathop{\mbox{\rm rank}}}
\newcommand{\ellipsoid}{{\mathcal E}}
\newcommand{\sround}{\mathop{\mbox{\rm SmartRound}}}
\newcommand{\round}{\mathop{\mbox{\rm round}}}
\newcommand{\myspan}{\mathop{\mbox{\rm span}}}
\newcommand{\edges}{\mathop{\mbox{\rm edges}}}
\newcommand{\conv}{\mathop{\mbox{\rm conv}}}
\newcommand{\sign}{\mathop{\mbox{\rm sign}}}
\newcommand{\poly}{\mathop{\mbox{\rm poly}}}

\newcommand{\ignore}[1]{}
\newcommand{\equaldef}{\stackrel{\text{\tiny def}}{=}}
\newcommand{\equaltri}{\triangleq}
\newcommand{\tr}{\ensuremath{{\scriptscriptstyle\mathsf{T}}}}
\def\trace{{\bf Tr}}
\def\dist{{\bf P}}
\def\mydist{{\bf Q}}
\def\reals{{\mathbb R}}
\newcommand{\E}{\mathop{\mbox{\bf E}}}
\newcommand\ball{\mathbb{B}}
\newcommand\var{\mbox{Var}}
\def\bzero{\mathbf{0}}

\def\Exp{\mathop{\mbox{E}}}
\def\mP{{\mathcal P}}
\def\risk{{\mbox{Risk}}}
\def\ent{{\mbox{ent}}}
\def\mG{{\mathcal G}}
\def\mA{{\mathcal A}}
\def\V{{\mathcal V}}
\def\lhat{\hat{\ell}}
\def\Lhat{\hat{L}}
\def\ltil{\tilde{\ell}}
\def\Ltil{\tilde{L}}

\def\Ahat{\hat{\mathbf{A}}}

\def\bold0{\mathbf{0}}
\def\boldone{\mathbf{1}}
\def\boldp{\mathbf{p}}
\def\boldq{\mathbf{q}}
\def\boldr{\mathbf{r}}
\def\boldA{\mathbf{A}}
\def\boldI{\mathbf{I}}
\newcommand\mycases[4] {{
\left\{
\begin{array}{ll}
    {#1}, & {#2} \\\\
    {#3}, & {#4}
\end{array}
\right. }}
\newcommand\mythreecases[6] {{
\left\{
\begin{array}{ll}
    {#1}, & {#2} \\\\
    {#3}, & {#4} \\\\
    {#5}, & {#6}
\end{array}
\right. }}
\def\bb{\mathbf{b}}
\def\bc{\mathbf{c}}
\def\bB{\mathbf{B}}
\def\bC{\mathbf{C}}
\def\be{\mathbf{e}}
\def\etil{\tilde{\mathbf{e}}}
\def\bk{\mathbf{k}}
\def\bu{\mathbf{u}}
\def\bx{\mathbf{x}}
\def\w{\mathbf{w}}
\def\by{\mathbf{y}}
\def\bz{\mathbf{z}}
\def\bp{\mathbf{p}}
\def\bq{\mathbf{q}}
\def\br{\mathbf{r}}
\def\bu{\mathbf{u}}
\def\bv{\mathbf{v}}

\def\bw{\mathbf{w}}
\def\ba{\mathbf{a}}
\def\bA{\mathbf{A}}
\def\bB{\mathbf{B}}
\def\bC{\mathbf{C}}
\def\bD{\mathbf{D}}
\def\bS{\mathbf{S}}
\def\bG{\mathbf{G}}
\def\bI{\mathbf{I}}
\def\bJ{\mathbf{J}}
\def\bP{\mathbf{P}}
\def\bQ{\mathbf{Q}}
\def\bV{\mathbf{V}}
\def\bone{\mathbf{1}}


\def\xhat{\hat{\mathbf{x}}}
\def\xbar{\bar{\mathbf{x}}}
\def\vol{\mbox{vol}}
\def\trace{{\bf Tr}}
\def\Exp{\mathop{\mbox{E}}}
\newcommand{\val}{\mathop{\mbox{\rm val}}}
\renewcommand{\deg}{\mathop{\mbox{\rm deg}}}
\newcommand{\eps}{\varepsilon}

% Macros for this chapter
\def\bone{\mathbf{1}}
\newcommand{\sphere}{\ensuremath{\mathbb {S}}}
\newcommand{\simplex}{\ensuremath{\Delta}}
\newcommand{\diag}{\mbox{diag}}
\newcommand{\K}{\ensuremath{\mathcal K}}
\newcommand{\Ktil}{\ensuremath{\mathcal{\tilde{K}}}}
\def\mA{{\mathcal A}}
\newcommand{\x}{\ensuremath{\mathbf x}}
\newcommand{\vn}{\ensuremath{\mathbf n}}
\newcommand{\rv}[1][t]{\ensuremath{\mathbf r_{#1}}}
\newcommand{\y}{\ensuremath{\mathbf y}}
\newcommand{\z}{\ensuremath{\mathbf z}}
\newcommand{\h}{\ensuremath{\mathbf h}}
\newcommand{\xtil}[1][t]{\ensuremath{\mathbf {\tilde{x}}_{#1}}}
\newcommand{\xv}[1][t]{\ensuremath{\mathbf x_{#1}}}
\newcommand{\xvh}[1][t]{\ensuremath{\mathbf {\hat x}}_{#1}}
\newcommand{\ftildeh}[1][t]{\ensuremath{\mathbf {\hat f}}_{#1}}
\newcommand{\xvbar}[1][t]{\ensuremath{\bar{\mathbf x}_{#1}}}
\newcommand{\xstar}{\ensuremath{\mathbf x^{*}}}
\newcommand{\xstarr}{\ensuremath{\mathbf x^{\star}}}
\newcommand{\yv}[1][t]{\ensuremath{\mathbf y_{#1}}}
\newcommand{\zv}[1][t]{\ensuremath{\mathbf z_{#1}}}
\newcommand{\fv}[1][t]{\ensuremath{\mathbf f_{#1}}}
\newcommand{\gv}[1][t]{\ensuremath{\mathbf g_{#1}}}
\newcommand{\qv}[1][t]{\ensuremath{\mathbf q_{#1}}}
\newcommand{\pv}[1][t]{\ensuremath{\mathbf p_{#1}}}
\newcommand{\ftilde}[1][t]{\ensuremath{\mathbf {\tilde f}_{#1}}}
\newcommand{\gtilde}[1][t]{\ensuremath{\mathbf {\tilde g}_{#1}}}
\newcommand{\ev}[1][i]{\ensuremath{\mathbf e_{#1}}}
\newcommand{\uv}{\ensuremath{\mathbf u}}
\newcommand{\vv}{\ensuremath{\mathbf v}}
\newcommand{\A}[1][t]{\ensuremath{\mathbf A_{#1}}}
\newcommand{\Ftilde}[1][t]{\ensuremath{\mathbf {\tilde F}_{#1}}}
\newcommand{\Gtilde}[1][t]{\ensuremath{\mathbf {\tilde G}_{#1}}}
\newcommand{\hessum}[1][t]{\ensuremath{\boldsymbol{H}_{1:#1}}}
\newcommand{\reg}[1][t]{\ensuremath{\lambda_{#1}}}
\newcommand{\regsum}[1][t]{\ensuremath{\boldsymbol{\lambda}_{1:#1}}}
\newcommand{\bM}{\ensuremath{\mathcal M}}
%\newcommand{\F}{\ensuremath{\mathcal F}}
\newcommand{\ntil}{\tilde{\nabla}}
\newcommand{\Ftil}{\tilde{F}}
\newcommand{\ftil}{\tilde{f}}
\newcommand{\rhotil}{\tilde{\rho}}
\newcommand{\gtil}{\tilde{g}}
\newcommand{\mutil}{\tilde{\mu}}
\newcommand{\Atil}{\tilde{A}}
\newcommand{\Qtil}{\tilde{Q}}
\newcommand{\vtil}{\tilde{v}}
\newcommand\super[1] {{^{(#1)}}}
\def\regret{\ensuremath{\mathrm{{regret}}}}
%\def\regretEM{\ensuremath{\text{\em regret}}}
\def\bb{\mathbf{b}}
\def\be{\mathbf{e}}
\def\bu{\mathbf{u}}
\def\bx{\mathbf{x}}
\def\by{\mathbf{y}}
\def\bz{\mathbf{z}}
\def\bp{\mathbf{p}}
\def\bdelta{\mathbf{\delta}}
\def\bq{\mathbf{q}}
\def\br{\mathbf{r}}
\def\bv{\mathbf{v}}
\def\ba{\mathbf{a}}
\def\bI{\mathbf{I}}
\def\bJ{\mathbf{J}}
\def\bP{\mathbf{P}}
\def\bV{\mathbf{V}}
\def\sgn{\text{sgn}}
\def\eps{\varepsilon}
\def\epsilon{\varepsilon}
\def\tsum{{\textstyle \sum}}

%\def\ogd{{\sc Online Gradient Descent}\xspace}
\def\ogd{{online gradient descent}\xspace}
\def\R{\ensuremath{\mathcal R}}



\begin{document}

%%% make header - do not modify! 
\noindent
\begin{minipage}[t]{1\columnwidth}%
\textsc{topological methods in engineering}\hspace*{\fill}048985
\vspace{2mm}

\textsc{\LARGE robust topological inference}\hspace*{\fill}\textsc{\lecturedate}

\noindent \rule[0.5ex]{1\linewidth}{1pt}

\textsc{tom norman}
\vspace{10mm}
\end{minipage}
%%%



\newcommand {\image}[5] {
    \begin{figure}
        \begin{center}
		\includegraphics[width=#4\textwidth,height=#3\textheight]{imgs/#1}
		%\includegraphics[width=\textwidth,height=#3\textheight,keepaspectratio]{imgs/#1}
		\caption{#2}
		\label{#5}
        \end{center}
    \end{figure}
}

%%%%%%%%%%%%%%%%%%%%%%%%%%%%%%%%%%%%%%%%%%%%%%%%%%%%%%%%%%%%%%%%
%% BODY OF SCRIBE NOTES GOES HERE
%%%%%%%%%%%%%%%%%%%%%%%%%%%%%%%%%%%%%%%%%%%%%%%%%%%%%%%%%%%%%%%%

\section{Introduction}
	The goal of geometric inference on point cloud data is to recover the geometry of the underlying shape (i.e. reconstruction) or the topological features (i.e. Betti numbers).
	One way to do so is using distance functions, however one major drawback is lack of robustness to noise and outliers.
	In  this paper the authors introduce smooth distance-like function (DTM) based on probability measure that has topological guarantees even in the presence of outliers.\\
	Moreover when empirical distribution ($\frac{1}{n}$ for each data point) is chosen over the data, DTM can be easily and intuitively evaluated.
	They also show how to use bootstrap to test the significance of the topological features and also choose the function hyperparameters.\\
	The code for this summary is in \cite{git}.
\section{Background}

We'll define some key-concepts as distance function, sublevel sets and such in order to understand the motives to using DTM.

\subsection{Distance Function}
Let $S \subset \mathbb{R}^d$ be compact set. The distance function on $S$ is defined as
$$\Delta_S(x) = \inf_{y \in s} \norm{y-x}, \forall x \in S$$

\subsection{Sublevel Sets}
Sublevel set of $S$ defined as
$$L_t = \left\{ x: \Delta_S(x) \leq t \right\}, t: 0 \rightarrow \infty$$
We'll refer to $t$ as "time".\\
As $t$ gets larger the topology of $L_t$ changes: connected components, holes, etc appear and merge,\\ so for $s < t: L_s \subseteq L_t$.\\
Note that for $t \rightarrow \infty$ we get one big connected component.


\subsection{Persistent Homology}
The method to compute the topological features that appear and merge as $t$ gets larger.\\
When feature i appears it's assigned a birth time $b_i$ ($t$ for which it started to exist) and when it merges an older feature it's assigned a death time $d_i$.\\
Lifetime of feature i is defined as $|d_i - b_i|$.
Intuitively the bigger the lifetime of a feature the more relevant / persistent it is.

\subsection{Persistent Diagram}
A lifetime of a feature can be represented in the 2D plane as the coordinate ($b_i,d_i$).\\
The set of those coordinates for given set $S$ is called the persistence diagram of $\Delta_S$. Note that birth of a feature comes before his death so the diagram is contained above the diagonal.\\
As we discussed before, the more persistent a feature is, the bigger its lifetime thus further away it's from the diagonal (where birth equals death). Features close to the diagonal can be seen as topological noise.\\
Note that in all persistence diagrams there exists a 0-dimension feature (connected component) at coordinate $(0,\infty)$.

\image{three.pdf}{Plots of distance function, sublevel set and persistent diagram of two circles with different radii.}{0.3}{1}{fig:three}

\subsection{Bottleneck Distance}
Given 2 compact sets $S_1,S_2$ with corresponding distance functions $\Delta_1, \Delta_2$ and diagrams $D_1,D_2$ (including the diagonal), define the bottleneck distance between them as
$$W_\infty (D_1,D_2) = \min_{g: D_1 \rightarrow D_2} \sup_{z \in D_1} \norm{z - g(z)}_\infty$$
$$g \text{ is a bijection.}$$
In words: the bottleneck distance is the maximum $L_\infty$ distance between points of $D_1,D_2$ after minimizing over all pairing of the points (including the diagonals).\\
A property of persistence diagram is stability; according to the Persistence Stability Theorem (Cohen-Steiner et al. (2005); Chazal et al. (2012)):
$$W_\infty (D_1,D_2) \leq \norm{\Delta_1 - \Delta_2}_\infty$$
we'll use it later when discussing significance.



\image{bottleneck_distance_example.png}{The bottleneck distance is the longest red edge.}{0.3}{0.4}{fig:bottleneck}

\section{Motivation}

Given sample $X_1,...,X_2 \sim P$ the empirical distance function is defined as
$$\hat{\Delta}(x) = \min_{X_i} \norm{x-X_i}$$

\begin{lemma}
	Suppose that $P$ is supported on $S$ and has a density bounded away from 0 and $\infty$, then
	$$\sup_{x} |\hat{\Delta}(x) - \Delta_S(x)| \overset{P}{\rightarrow} 0$$
\end{lemma}
\text{ }\\
This lemma justifies using $\hat{\Delta}$ to estimate persistence homology of sublevel sets of $\Delta_S$. In fact the sublevel sets of $\hat{\Delta}$ are balls around the data points with raduis t:
$$L_t = \left\{ x: \hat{\Delta} (x) \leq t \right\} = \overset{n}{\underset{i=1}{\cup}} B_{x_i}(t)$$
which is the same as Cech complex, the persistent homology will then be extracted using filtration and linear algebra.\\
However, if the data includes noise or ouliers the empirical distance function $\hat\Delta$ no longer represents $\Delta_S$ truly (see Figure \ref{fig:noise}). The paper introduces a new notion of distance- distance to (probability) measure to overcome the noise/ouliers problem.

\image{noise.pdf}{1000 points samples from a circle + 200 points sampled from gaussian RV. We can see that the normal distance function (middle) creates lots of features while DTM (right) removes most of them and leaves one that looks significant}{0.3}{1}{fig:noise}

\section{Distance to (Probability) Measure}
Given probability measure $P$, for $m \in (0,1)$ define DTM as
$$DTM^2_{P,m}(x) = \frac{1}{m} \int_{0}^{m} \left( G_x^{-1}(u)\right)^2 du$$
where $G_x(t) = \mathbb{P} \left( \norm{X - x} \leq t \right)$.\\
When considering $P$ as the empirical distribution $\hat P$ (i.e. for n data points give each $\frac{1}{n}$ probability mass), we get
$$G_x(t) = \frac{\# \left\{X_i: X_i \in B_x(t) \right\}}{n}$$
$$\Rightarrow G_x^{-1}(u) \text{ will give us the distance of the farthest $X_i$ from $x$ s.t. } \mathbb{P} \left( \norm{X-x} \leq \norm{X_i - x} \right) = u$$
Define $k=\lceil u * n \rceil$. Because $G_x^{-1} (u)$ is constant in the range $u \in (\frac{k}{n}, \frac{k+1}{n}]$, then $G_x^{-1}(u)$ becomes a step function with step width $\frac{1}{n}$ and the value of each step is the distance between $x$ and its k-th nearest neighbor, which changes the integral to discrete sum of k nearest neighbors:
$$DTM^2_{X,k} (x) =\frac{1}{k} \sum_{X_i \in kNN(x)} \norm{X_i - x}^2$$
Note that for k=1 we get the regular distance function.


\section{Significance of Topological Features}
\subsection{Bootstrapping}
We would like to get statistical estimations about our data like mean, variance and confidence intervals. Because the population (in our case the underlying shape) is unknown we will instead compute those estimation on our sampled data (the point cloud) by resampling with replacement from the sampled data number of times and creating an empirical distribution. This distribution is asymptotically consistent with the population distribution.\\
$\text{ }$\\
\begin{algorithm}[H]
\caption{Bootstrapping}
\SetAlgoLined
\SetKwInOut{Input}{Input}
\Input{
	$X = \left\{ x_i \right\}_{i=1}^n$ - point cloud data\\
	$k \in \left[1,n-1\right]$ - smoothing parameter\\
	$B$ - number of resamples\\
	$\alpha$ - significance level
}
Compute $DTM_{X,k}(x), \forall x \in S$\\
\For{j = 1:B}{
	Draw n samples from $X$ with replacement, denoted $X_j$\\
	Compute $\theta_j = \sqrt{n} \norm{DTM_{X,k}(x) - DTM_{X_j,k}(x)}_\infty$
}
Compute $\hat t_\alpha = \underset{t}{\inf} \left(\sum_{j=1}^B \textbf{1}_{\{\theta_j \geq t\}} \leq \alpha B \right)$\\

\KwResult{$\hat t_\alpha$ - confidence band}
\end{algorithm}

\text{ }\\
\text{ }\\
In theory, a feature with birth and death time $(b_i,d_i)$ is said to be significant if $|d_i - b_i| > 2\frac{t_\alpha}{\sqrt{n}}$ where $t_\alpha$ is defined by
$$\mathbb{P} \left( \sqrt{n} \norm{DTM_{X,k}(x) - DTM_{P,m}(x)}_\infty > t_\alpha \right) = \alpha$$
When using empirical DTM, $t_\alpha$ can be estimated by bootstrapping to get $\hat t_\alpha$ which is defined by
$$\mathbb{P} \left( \sqrt{n} \norm{DTM_{X,k}(x) - DTM_{X_j,k}(x)}_\infty > \hat t_\alpha | x_1,x_2,...,x_n \right) = \alpha$$

\text{ }\\
To see why this makes sense let $\mathcal{D}$ be the set of persistence diagrams. Let $D$ be the true diagram of the underlying shape and let $\hat D$ be the estimated diagram. Let
$$\mathcal{L}_n = \left\{ E \in \mathcal{D} : W_\infty(\hat D, E) \leq \frac{\hat t_\alpha}{\sqrt{n}} \right\}$$
be the set of all diagrams that are $\frac{\hat t_\alpha}{\sqrt{n}}$ away from $\hat D$.\\
From persistence diagram stability we get Theorem 3.5 (in the paper)
$$W_\infty (D_P,D_Q) \geq \norm{DTM_{P,m}(x) - DTM_{Q,m}(x)}_\infty$$

\text{ }\\
Taking $Q$ as the empirical distribution of $X$ we get
$$\mathbb{P} \left( D \in \mathcal{L}_n \right) = \mathbb{P} \left( W_\infty (D, \hat D) \leq \frac{\hat t_\alpha}{\sqrt{n}} \right) \geq \mathbb{P} \left( \sqrt{n} \norm{DTM_{P,m}(x) - DTM_{X,k}(x)}_\infty \leq \hat t_\alpha \right) \underset{n\rightarrow \infty}{\rightarrow} 1-\alpha$$

\text{ }\\
So $|d_i - b_i| > 2\frac{\hat t_\alpha}{\sqrt{n}} \Leftrightarrow$ the feature cannot be matched to the diagonal for any diagram in $\mathcal{L}_n$.\\
The diagonal represents features with 0 lifetime, so for $|d_i - b_i| \leq 2\frac{\hat t_\alpha}{\sqrt{n}}$ the feature isn't different from 0 lifetime feature (i.e. topological noise), with $1-\alpha$ probability.

\def\tha{\hat t_\alpha}
\def\ftn{2\frac{\tha}{\sqrt{n}}}
\section{Choosing Smoothing Parameter}
Using a smoothing parameter rises the need for hyper-parameter tuning. So what is the metric for which a smoothing parameter considered good?\\
The paper suggests two quantities to measure the amount of significance information.\\
Define $\ell_i(m)$ as the lifetime of feature $i$ with parameter $m$.

Number of features that are $\alpha$ significant:
$$N(m) = \# \left\{ i : \ell_i(m) > \ftn \right\}$$

Total $\alpha$ significant persistence:
$$S(m) = \sum_i \left[ \ell_i(m) - \ftn \right]_+$$

\text{ }\\
The $m$ that maximizes $N(m)$ or $S(m)$ is chosen (see Figure \ref{fig:bootstrap}).

\image{boot.pdf}{Using the same noised data from Figure \ref{fig:noise}, $k=[1:20:201], B=30, \alpha=0.05.$\\
$m\geq 41/1200$ will maximize N and $m=201/1200$ will maximize S.}{0.5}{1}{fig:bootstrap}


\section{Criticism}

The paper is very comprehensive with respect to the analysis of the proposed method. Most of the theorems and conclusions are made for the general DTM (with general distribution $P$, not just the empirical $\hat P$).
The empirical DTM is very intuitive- take the distance to a cluster of data points to smooth the noise, also the way to choose smoothing parameters is easy to understand.\\
The authors used mainly synthetic examples which doesn't show if the method is really applicative, however they implemented almost everything in R, and made it open-source so we can try on real data. As a small bonus, in the end, they provide some points for further research.


\newcommand\figu[1]{Figure \ref{#1}}
\section{Experiments}
There is R package called 'TDA' that contains almost all of the functions from the paper.
My first step was to install everything on a docker then learn the basics of R using the tutorial \cite{tut}. After I got the basics I started experimenting how robust DTM is, and finally I tried to classify numbers on MNIST using the package.


\subsection{Unit Circle}

\image{u_breaking.pdf}{Adding $[30:30:330]$ samples from $U[-2,2]^3$}{0.7}{1}{fig:uninum}
\image{n_breaking.pdf}{Adding $[50:100:550]$ samples from $\mathcal{N}(0,1)$}{0.7}{1}{fig:num}
\image{s_breaking.pdf}{Adding 10 samples from $\mathcal{N}(0,[1.5:1:6.5])$}{0.7}{1}{fig:sigma}

In this sections I used 1000 samples from the unit circle and $k = [1, 21, 41]$.\\
(The following figures are of the same idea as \figu{fig:bootstrap}, showing the persistence diagram given k for multiple ks)\\
\figu{fig:uninum} shows the unit circle with uniform noise as the number of samples varies.\\
\figu{fig:num} shows the unit circle with gaussian noise as the number of samples varies.\\
\figu{fig:sigma} shows the unit circle with 10 samples from a gaussian noise as the standard deviation varies.\\
As we can see all of tests break DTM pretty fast, the best situation is when the noise is not very 'spread' as with $\mathcal{N}(0,1)$ (\figu{fig:num}).


\subsection{Multiple Shapes}

\image{sc_data.pdf}{Disjoint shapes}{0.5}{1}{fig:disjoint}
\image{cc_data.pdf}{Joint shapes}{0.5}{1}{fig:joint}

I combined several disjointed shapes (a torus, a circle inside a circle and a sphere) and computed the persistent diagram, then added gaussian noise and computed the persistent digaram, \figu{fig:disjoint}.\\
Next I merged the shapes and recomputed the persistent diagrams, \figu{fig:joint}.\\
(There are gifs on github to show those shapes \cite{git}).\\
With both cases (disjoint and joint) DTM gave nice results, it removed most of the topological noise (features close to the diagonal) and kept the significant features (2 voids and 4 holes, and in the disjoint case an additional 3 connected components).


\subsection{Shape Reconstruction}

\image{yoda.png}{Reconstruction of Yoda from noisy version}{0.5}{1}{fig:yoda}

Another option for using DTM is its ability to smooth noise from point cloud of a shape using a grid and some threshold on the DTM value for each point on the grid. I used PointCleanNet dataset \cite{point} to reconstruct Yoda's face from its noisy version (\figu{fig:yoda}), but it's possible for every shape in this dataset.


\subsection{MNIST Point Cloud}
I'll try to classify 0 vs 6 using DTM.
\begin{enumerate}
	\item
		Take only numbers 0 and 6 examples and split to train/test
	\item
		For each number iterate over all given examples, for each example run DTM with k=[1,2,5,10] and extract the hole with the maximum lifetime.\\
		Compute the mean of those lifetimes.
	\item
		Iterate over all given examples in the test data, if the lifetime of the most significant hole is closer to the mean of 0, classify as 0, else as 6.\\
		If no hole or the lifetime is equally close, don't classify.
\end{enumerate}
I made this experiment also for 0 vs 9 and 6 vs 9.
The results: (cont. on page 14)
\begin{itemize}
	\item
		\underline{0 vs 6}\\
Trained on 11841 examples\\
Tested on  1938 examples\\
k = 1\\
0 mean = 3.65642875028104 | 6 mean = 1.29528674239364\\
Classified correctly 1734 examples\\
Not classified correctly 179 examples\\
Not classified 25 examples\\
k = 2\\
0 mean = 3.25105078064842 | 6 mean = 0.924533413511397\\
Classified correctly 1761 examples\\
Not classified correctly 159 examples\\
Not classified 18 examples\\
k = 5\\
0 mean = 3.19517496139027 | 6 mean = 0.87762722292563\\
Classified correctly 1751 examples\\
Not classified correctly 161 examples\\
Not classified 26 examples\\
k = 10\\
0 mean = 3.01921602751145 | 6 mean = 0.741459643826337\\
Classified correctly 1735 examples\\
Not classified correctly 151 examples\\
Not classified 52 examples

	\item
		\underline{0 vs 9}\\
Trained on 11872 exmaples\\
Tested on  1989 examples\\
k = 1\\
0 mean = 3.65642875028104 | 9 mean = 1.71784789743753\\
Classified correctly 1700 examples\\
Not classified correctly 282 examples\\
Not classified 7 examples\\
k = 2\\
0 mean = 3.25105078064842 | 9 mean = 1.37118797421732\\
Classified correctly 1720 examples\\
Not classified correctly 263 examples\\
Not classified 6 examples\\
k = 5\\
0 mean = 3.19517496139027 | 9 mean = 1.32001191988419\\
Classified correctly 1718 examples\\
Not classified correctly 264 examples\\
Not classified 7 examples\\
k = 10\\
0 mean = 3.01921602751145 | 9 mean = 1.13703358223009\\
Classified correctly 1738 examples\\
Not classified correctly 236 examples\\
Not classified 15 examples

	\item
		\underline{6 vs 9}\\
Trained on 11867 examples\\
Tested on  1967 examples\\
k = 1\\
6 mean = 1.29528674239364 | 9 mean = 1.71784789743753\\
Classified correctly 1190 examples\\
Not classified correctly 745 examples\\
Not classified 32 examples\\
k = 2\\
6 mean = 0.924533413511397 | 9 mean = 1.37118797421732\\
Classified correctly 1241 examples\\
Not classified correctly 702 examples\\
Not classified 24 examples\\
k = 5\\
6 mean = 0.87762722292563 | 9 mean = 1.32001191988419\\
Classified correctly 1228 examples\\
Not classified correctly 706 examples\\
Not classified 33 examples\\
k = 10\\
6 mean = 0.741459643826337 | 9 mean = 1.13703358223009\\
Classified correctly 1202 examples\\
Not classified correctly 698 examples\\
Not classified 67 examples
\end{itemize}

As we can infer from the results DTM with k=5 works best on MNIST, while with k=10 we get poor results.
For 0 vs 6/9 we get good results because the hole in zero is more significant (as we can see from the difference in the means).\\
Overall DTM with reasonable amount of tuning outperforms the regular distance function.

\text{}\\
Dataset taken from \cite{mnist}.








\begin{thebibliography}{9}
\bibitem{dtm}
Chazal, Frédéric \& Cohen-Steiner, David \& Mérigot, Quentin. (2011). Geometric Inference for Probability Measures. Foundations of Computational Mathematics. 11. 733-751. 10.1007/s10208-011-9098-0.

\bibitem{rdtm}
Chazal, Frédéric \& Fasy, Brittany \& Lecci, Fabrizio \& Michel, Bertrand \& Rinaldo, Alessandro \& Wasserman, Larry. (2014). Robust Topological Inference: Distance To a Measure and Kernel Distance. Journal of Machine Learning Research. 18.

\bibitem{tut}
Fasy, Brittany Terese, Jisu Kim, Fabrizio Lecci and Clément Maria. Introduction to the R package TDA. ArXiv abs/1411.1830 (2014): n. pag.

\bibitem{git}
\url{https://github.com/tomnorman/dtm}

\bibitem{point}
Rakotosaona, Marie-Julie and La Barbera, Vittorio and Guerrero, Paul and Mitra, Niloy J and Ovsjanikov, Maks. PointCleanNet: Learning to Denoise and Remove Outliers from Dense Point Clouds. Computer Graphics Forum. 2019.

\bibitem{mnist}
\url{https://www.kaggle.com/cristiangarcia/pointcloudmnist2d}

\end{thebibliography}
%%%%%%%%%%%%%%%%%%%%%%%%%%%%%%%%%%%%%%%%%%%%%%%%%%%%%%%%%%%%%%%%

\end{document}
